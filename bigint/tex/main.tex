\documentclass[a4paper,12pt]{article}

\usepackage{cmap}
\usepackage[T2A]{fontenc}
\usepackage[utf8]{inputenc}
\usepackage[english,russian]{babel}
\usepackage{amsthm}
\usepackage{amssymb}
\usepackage{amsmath}
\usepackage{mathtools}
\usepackage{indentfirst}
\usepackage{fullpage}
\usepackage{multicol}
\usepackage{listings}

\date{}
\author{}

\begin{document}
\textbf{Ввод:} $n$-значное (делимое) и $m$-значное (делитель) числа.

\textbf{Вывод:} результат деления первого аргумента на второй. \\

Запишем рассматриваемый алгоритм в псевдокоде:
\begin{lstlisting}[language=Python]
def div(divisible: Number, divisor: Number) -> Number:
    if divisor == 0:
        throw DivisionByZeroException()
    
    res = []
    while len(divisible) > 0:
        offset = None
        for i in [1, len(divisible)]:
            if divisible[:i] >= divisor:
                offset = i
                break

        if offset is None:
            break

        cur = 0
        while divisible[:offset] > divisor:
            divisible[:offset] -= divisor
            cur += 1

        res.append(cur)
    return res
\end{lstlisting}

В худшем случае внешний цикл совершит $n$ итераций, первый вложенный "--- $m$, второй вложенный "--- $B$, где $B$ "--- основание системы счисления, константа.
На каждой итерации внутренних циклов совершаются только элементарные операции.

Значит в худшем случае $T(n, m) = n(m + B)$.
$T(n, m) = O(n \cdot m)$, так как $nm > nB$ при $n, m \to \infty$

В лучшем же случае оба аргумента равны, тогда внешний цикл совершит ровно одну итерацию, первый вложенный "--- $n$, второй вложенный "--- тоже ровно одну. Тогда получаем $T(n, m) = T(n, n) = O(n)$.
\end{document}
